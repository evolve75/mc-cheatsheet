% mc-cheatsheet.tex
% Midnight Commander (mc) — 4 logical pages (US Letter)
% Booklet imposition (2-up, short-edge duplex) is done afterwards via pdfbook2.
%
% Build:
%   latexmk -pdf -xelatex mc-cheatsheet.tex
%
% Booklet:
%   pdfbook2 --paper=letterpaper --short-edge mc-cheatsheet.pdf
%
% Primary sources (see References on last page):
%   - mc(1) manual: https://source.midnight-commander.org/man/mc.html
%   - Gist cheat sheet: https://gist.github.com/samiraguiar/9cd4264445545cfd459d
%   - Project site: https://midnight-commander.org/
%
% vim: set ft=tex :

\documentclass[9pt,letterpaper]{article}

% ---------- Page geometry (lamination-safe) ----------
\usepackage[
  left=0.50in,
  right=0.50in,
  top=0.75in,
  bottom=0.75in
]{geometry}

% ---------- Fonts ----------
\usepackage{fontspec}
\defaultfontfeatures{Ligatures=TeX}
\IfFontExistsTF{TeX Gyre Pagella}{\setmainfont{TeX Gyre Pagella}}{\setmainfont{DejaVu Serif}}
\IfFontExistsTF{TeX Gyre Heros}{\setsansfont{TeX Gyre Heros}}{\setsansfont{DejaVu Sans}}
\IfFontExistsTF{DejaVu Sans Mono}{\setmonofont{DejaVu Sans Mono}}{\setmonofont{Latin Modern Mono}}

% ---------- Layout ----------
\usepackage{multicol}
\setlength{\columnsep}{0.22in}
\raggedcolumns{}

% ---------- Tables ----------
\usepackage{booktabs}
\usepackage{array}
\usepackage{xcolor}

% ---------- Lists ----------
\usepackage{enumitem}
\setlist[itemize]{leftmargin=*, itemsep=0.12em, topsep=0.12em}

% ---------- Links / PDF outlines ----------
\usepackage{bookmark}
\usepackage{hyperref}
\usepackage{xurl}
\hypersetup{
  colorlinks=true,
  linkcolor=blue,
  urlcolor=blue,
  pdftitle={Midnight Commander (mc) Cheatsheet},
  pdfsubject={Key bindings and quick reference},
  pdfkeywords={midnight commander, mc, file manager, cheatsheet}
}

% ---------- Header / footer ----------
\usepackage{fancyhdr}
\pagestyle{fancy}
\fancyhf{}
\renewcommand{\headrulewidth}{0.3pt}
\renewcommand{\footrulewidth}{0.3pt}
\fancyhead[L]{\textsf{Midnight Commander (mc)}}
\fancyhead[R]{\textsf{Cheatsheet}}
\fancyfoot[R]{\textsf{Page \thepage\ of 4}}

% ---------- House style ----------
\setlength{\parindent}{0pt}
\setlength{\parskip}{0.24em}
\sloppy
\emergencystretch=1.2em

% ---------- Macros ----------
\newcommand{\app}{\textsf{Midnight Commander}}
\newcommand{\key}[1]{\texttt{#1}}
\newcommand{\ks}[1]{\textbf{\texttt{#1}}}
\newcommand{\sect}[1]{\textsf{\textbf{#1}}}
\newcommand{\sectionrule}{\vspace{0.14em}\hrule\vspace{0.30em}}
\newcommand{\tinyurl}[1]{\texttt{\small\url{#1}}}
\newcommand{\pathtt}[1]{\texttt{\path{#1}}}

% Literal specials (safe in tables)
\newcommand{\pct}{\char`\%}
\newcommand{\lbracechar}{\char`\{}
\newcommand{\rbracechar}{\char`\}}

% Wrap-friendly code for long protocol-ish strings
% Commands/snippets: preserve literal spaces
\newcommand{\codewrap}[1]{{\ttfamily\small #1}\space}
% Path/protocol strings: allow breaks at / : . _ - etc., and preserve trailing space
\newcommand{\codepath}[1]{{\ttfamily\small\path{#1}}\space}

% ---------- Key table environment ----------
\newcommand{\keycolw}{0.24\linewidth}
\newcommand{\actcolw}{0.72\linewidth}
\newcommand{\aftertable}{\vspace{0.48em}}
\newenvironment{keytable}
{%
  \vspace{0.22em}%
  \begin{tabular}{@{} >{\raggedright\arraybackslash}p{\keycolw} >{\raggedright\arraybackslash}p{\actcolw} @{}}
  \toprule
  \textsf{Key} & \textsf{Action}\\
  \midrule
}
{%
  \bottomrule
  \end{tabular}
  \aftertable{}
}

% chktex 36
\newcommand{\manref}[2]{\mbox{#1(#2)}}

\begin{document}

% ===================== PAGE 1 =====================
{\Large\textsf{\textbf{\app{} (mc) — one-stop cheatsheet}}}\par
\small

\medskip{}
{\raggedright{}
\textsf{Midnight Commander is an orthodox (two-pane) terminal file manager—think “visual shell”:
navigate directories, tag groups of files, and perform copy/move/rename/delete at speed, while
also having built-in viewer/editor, a configurable user menu, background jobs, and a Virtual File
System (VFS) that lets you browse archives and remote hosts (for example, \key{sftp://} and \key{sh://})
as if they were local directories.}\par
}
\medskip{}

\sectionrule{}

\begin{multicols}{2}

\sect{Legend / conventions}\par
\begin{itemize}
  \item \ks{C-x} = Ctrl+\key{x}. \quad \ks{A-x} = Alt+\key{x}.
  \item No Alt/Meta? Press \ks{Esc}, release, then type the character.
  \item Many terminals map \ks{Esc}+\key{1\ldots{}9,0} \(\rightarrow{}\) \ks{F1\ldots{}F10}.
  \item \ks{Esc Esc} quickly dismisses menus and popups — gist.
\end{itemize}

\sect{Core model}\par
\begin{itemize}
  \item \textbf{Current panel} + \textbf{other panel} (destination default).
  \item \textbf{Menu bar} (focus with \ks{F9}); \textbf{command line} at bottom.
  \item \textbf{Tagged files}: if any are tagged, most operations apply to the tagged set.
\end{itemize}

\sect{Launch \& CLI options — \manref{mc}{1}}\par
\begin{itemize}
  \item Start: \ks{mc} \; or \ks{mc dir1 [dir2]} (set panel dirs).
  \item Viewer: \ks{mc -v FILE} \quad Editor: \ks{mc -e [FILE]}.
  \item Subshell: \ks{mc -u} (disable) \; \ks{mc -U} (force).
  \item Color: \ks{mc -c} (force) \; \ks{mc -b} (B/W) \; \ks{mc -C ARG} (palette).
  \item Mouse/tty: \ks{mc -d} (no mouse) \; \ks{mc -g} (oldmouse) \; \ks{mc -x} (xterm).
  \item Keymap: \ks{mc -K FILE} \; \ks{mc --nokeymap}.
  \item Directory print: \ks{mc -P FILE} (for \key{mc.sh} “cd after exit” wrapper).
  \item Info: \ks{mc -V} (version) \; \ks{mc -h} (help).
\end{itemize}

\sect{Core function keys — File menu}\par
\begin{keytable}
\ks{F1} & Help (contextual).\\
\ks{F2} & User menu (custom actions and macros).\\
\ks{F3} & View file (\ks{Shift-F3} raw view; disregard extension).\\
\ks{F4} & Edit file (\ks{Shift-F4} create new file).\\
\ks{F5} & Copy selected or tagged to other panel directory.\\
\ks{F6} & Move selected or tagged (\ks{Shift-F6} rename under cursor).\\
\ks{F7} & Mkdir.\\
\ks{F8} & Delete.\\
\ks{F9} & Focus top menu bar.\\
\ks{F10} & Quit.\\
\end{keytable}

\sect{Selection — gist}\par
\begin{keytable}
\ks{Insert}/\ks{C-t} & Select or deselect file.\\
\ks{*} & Invert selection on files.\\
\ks{+} & Selection options (including custom pattern).\\
\ks{-} & Deselect options (pattern).\\
\end{keytable}

\sect{Quick navigation — gist}\par
\begin{keytable}
\ks{Tab}/\ks{C-i} & Jump to other panel.\\
\ks{A-c} & Quick cd dialog.\\
\ks{A-?} & Search dialog.\\
\ks{C-s} & Search for item.\\
\ks{A-s} & Incremental search (repeat \ks{A-s} to next occurrence).\\
\ks{C-\textbackslash} & Directory hotlist.\\
\end{keytable}

\end{multicols}

\newpage

% ===================== PAGE 2 =====================
{\Large\textsf{\textbf{Navigation, display, panels menu, and command prompt}}}\par
\small
\sectionrule{}

\begin{multicols}{2}

\sect{Navigation keys — gist}\par
\begin{keytable}
\ks{C-p}/\ks{Up} & Move selection to previous entry.\\
\ks{C-n}/\ks{Down} & Move selection to next entry.\\
\ks{A-g} & First visible item.\\
\ks{A-r} & Middle item.\\
\ks{A-j} & Last visible item.\\
\ks{A-v}/\ks{PgUp} & One page up.\\
\ks{A-p}/\ks{PgDn} & One page down.\\
\ks{A-<}/\ks{Home} & Top (first entry).\\
\ks{A->}/\ks{End} & Bottom (last entry).\\
\end{keytable}

\sect{Directory history — gist}\par
\begin{keytable}
\ks{A-y} & Previous directory in directory history.\\
\ks{A-u} & Next directory in directory history.\\
\ks{A-Shift-h} & Show path history.\\
\end{keytable}

\sect{Display — gist}\par
\begin{keytable}
\ks{C-r} & Refresh current panel.\\
\ks{C-u} & Swap panels.\\
\ks{A-,} & Toggle panel layout (horizontal/vertical).\\
\ks{C-x i} & Toggle other panel to information mode.\\
\ks{C-x q} & Toggle other panel to quick view mode.\\
\ks{A-i} & Other panel shows same directory as current.\\
\ks{A-o} & Show highlighted directory in the other panel.\\
\ks{A-t} & Change panel view (full/brief/long).\\
\ks{A-.} & Toggle “Show hidden files”.\\
\end{keytable}

\sect{Panels menu — via F9 then Left/Right}\par
\begin{keytable}
\ks{(menu)} & Listing mode: Full / Brief / Long / User-defined.\\
\ks{(menu)} & Info, Quick view, Tree, Tree view.\\
\ks{(menu)} & Sort order: name/ext/time/size/inode; reverse toggle.\\
\ks{(menu)} & Filter (limit visible); Select group (tag by pattern).\\
\ks{(menu)} & Rescan/reread directory; Panelize (fill from command output).\\
\end{keytable}

\columnbreak{}

\sect{Menus at a glance — F9}\par
\begin{keytable}
\ks{Left/Right} & Listing mode, sort, filter/select, quick view/info, tree, panelize, hotlist.\\
\ks{File} & View/edit/copy/move/mkdir/delete, link tools, quit.\\
\ks{Command} & Find file, quick cd, hotlist, compare directories, external panelize, jobs.\\
\ks{Options} & Configuration, appearance/layout, key learning, confirmations (build dependent).\\
\end{keytable}

\sect{File and directory operations — gist}\par
\begin{keytable}
\ks{C-x d} & Compare directories.\\
\ks{C-x c} & Chmod dialog.\\
\ks{C-x o} & Chown dialog.\\
\ks{C-x C-s} & Edit symlink.\\
\ks{C-x s} & Create symlink dialog.\\
\ks{C-x l} & Create hard link dialog.\\
\ks{C-x v} & Relative symlink tool on selected or tagged.\\
\ks{C-x a} & List active VFS directories.\\
\end{keytable}

\sect{Command prompt — gist}\par
\begin{keytable}
\ks{C-o} & Drop to the console (toggle panels).\\
\ks{A-Enter} & Put highlighted file name on command line.\\
\ks{C-x t} & Put selected or tagged item names on command line.\\
\ks{C-Shift-Enter} & Put full path of highlighted file on command line.\\
\ks{A-a}/\ks{C-x p} & Put full path of panel directory on command line.\\
\ks{A-h} & Show command history.\\
\ks{A-n}/\ks{A-p} & Move up or down through command history.\\
\ks{C-x \!} & External panelize (fill panel with command output).\\
\ks{C-x j} & Show background jobs.\\
\end{keytable}

\sect{Quit variants — gist}\par
\begin{keytable}
\ks{F10} & Quit (normal).\\
\ks{Shift-F10} & Quiet exit without confirmation.\\
\end{keytable}

\end{multicols}

\newpage

% ===================== PAGE 3 =====================
{\Large\textsf{\textbf{Viewer/editor, panelize recipes, user menu macros, keymaps}}}\par
\small
\sectionrule{}

\begin{multicols}{2}

\sect{File view — gist}\par
\begin{keytable}
\ks{C-f} & View the next file.\\
\ks{C-b} & View the previous file.\\
\end{keytable}

\sect{Viewer (mcview): practical extras}\par
\begin{keytable}
\ks{/} & Search.\\
\ks{n} & Next match.\\
\ks{Shift-n} & Previous match.\\
\ks{A-<}/\ks{A->} & Begin/end of file (common navigation).\\
\end{keytable}

\sect{Editor (mcedit): quick survival}\par
\begin{keytable}
\ks{F2} & Save.\\
\ks{F10} & Exit editor.\\
\ks{F7} & Search.\\
\ks{F4} & Replace.\\
\ks{C-y} & Delete line.\\
\ks{C-k} & Delete to end of line.\\
\end{keytable}

\sect{Panelize: “virtual panels” from commands}\par
\begin{keytable}
\ks{C-x \!} & External panelize (fill panel with command output).\\
\ks{Example} & \key{git ls-files} (repo file list).\\
\ks{Example} & \key{rg -l PATTERN} (files matching a pattern).\\
\ks{Example} & \key{find \. -type f -size +100M -print} (large files).\\
\ks{Example} & \key{ls -1 | \ldots{}} (pipelines that output one file per line).\\
\end{keytable}

\sect{Background jobs — gist}\par
\begin{keytable}
\ks{C-x j} & Jobs list; inspect or cancel background operations.\\
\end{keytable}

\columnbreak{}

\sect{User menu “@” commands — gist}\par
\begin{keytable}
\ks{F2-@} & Run a command on the currently highlighted item.\\
\ks{Example} & \ks{F2-@ unzip} — unzip selected file.\\
\ks{Example} & \ks{F2-@ zip -r foo.zip} — zip current directory as \key{foo.zip}.\\
\ks{Example} & \ks{F2-@ 7za x} — extract selected file with 7zip.\\
\ks{Example} & \ks{F2-@ 7za a foo.7z} — 7zip current directory as \key{foo.7z}.\\
\ks{File} & User menu file (common): \pathtt{~/.config/mc/menu}.\\
\end{keytable}

\sect{User menu macros — \manref{mc}{1}}\par
\begin{keytable}
\ks{\%f}, \ks{\%p} & Current file (name/path).\\
\ks{\%d}, \ks{\%D} & Current directory / other panel directory.\\
\ks{\%t}, \ks{\%T} & Tagged files (this/other panel).\\
\ks{\%v}, \ks{\%V} & Tagged files with full paths.\\
\ks{\%s}, \ks{\%S} & Selected files (tagged if any else current).\\
\ks{\%cd} & Change directory (useful for VFS entrypoints).\\
\ks{\%view[args]} & Internal viewer (ascii/hex/nroff/\ldots{}).\\
\textbf{\texttt{\pct{}\lbracechar{} prompt\rbracechar{}}} & Prompt user for substitution text.\\
\textbf{\texttt{\pct{} var\lbracechar{} ENV:default\rbracechar{}}} & Env var substitution with default.\\
\end{keytable}

\sect{Keymaps and Learn keys — \manref{mc}{1}}\par
\begin{keytable}
\ks{Learn keys} & Stores terminal sequences in \pathtt{~/.config/mc/ini} under \key{[terminal:TERM]}.\\
\ks{-K}/\ks{--keymap} & Load a specific keymap file.\\
\ks{MC\_KEYMAP} & Environment variable that selects a keymap.\\
\ks{\pathtt{~/.config/mc/mc.keymap}} & Common user keymap location.\\
\end{keytable}

\end{multicols}

\newpage

% ===================== PAGE 4 =====================
{\Large\textsf{\textbf{VFS, configuration paths, subshell, troubleshooting, references}}}\par
\small
\sectionrule{}

\begin{multicols}{2}

\sect{VFS:\@ browse “special paths” — \manref{mc}{1}}\par{}
\begin{keytable}
\ks{Concept} & VFS lets you \emph{cd into} archives and remotes like directories.\\
\ks{C-x a} & List active VFS directories — gist.\\
\ks{Tip} & VFS can stack (for example, an archive stored on an \key{sftp://} path).\\
\end{keytable}

\sect{Remote VFS — \manref{mc}{1}}\par{}
\begin{keytable}
\ks{sh://} & \codepath{sh://[user@]machine[:port]/[remote-dir]}\\
\ks{sftp://} & \codepath{sftp://[user@]machine[:port]/[remote-dir]}\\
\ks{Example} & \codepath{sh://joe@somehost:2222/private}\\
\ks{Example} & \codepath{sftp://joe@somehost:2222/private}\\
\end{keytable}

\sect{Archives and EXTfs — \manref{mc}{1}}\par{}
\begin{keytable}
\ks{Example} & \codewrap{cd documents.zip/uzip://} (man page example).\\
\ks{Pattern} & \codewrap{cd file.tar.gz/utar://} (extfs name varies).\\
\ks{Standalone} & \codewrap{cd fsname://} (enter an extfs “filesystem”).\\
\end{keytable}

\sect{Undelete (optional ext2) — \manref{mc}{1}}\par{}
\begin{keytable}
\ks{undel://} & \codewrap{undel://DEVICEPART} (for example, \codewrap{undel://sda2}) if built with support.\\
\end{keytable}

\columnbreak{}

\sect{Configuration paths — \manref{mc}{1}}\par{}
\begin{keytable}
\ks{Main config} & \pathtt{~/.config/mc/ini}\\
\ks{Panels} & \pathtt{~/.config/mc/panels.ini}\\
\ks{Extensions} & \pathtt{~/.config/mc/mc.ext.ini}\\
\ks{User menu} & \pathtt{~/.config/mc/menu}\\
\ks{Keymap} & \pathtt{~/.config/mc/mc.keymap}\\
\ks{Highlighting} & \pathtt{~/.config/mc/filehighlight.ini} (if supported).\\
\end{keytable}

\sect{Subshell notes — \manref{mc}{1}}\par{}
\begin{keytable}
\ks{-u / -U} & Disable / force subshell.\\
\ks{Concept} & Subshell shares your shell environment and aliases while mc runs.\\
\ks{bashrc} & Often \pathtt{~/.local/share/mc/bashrc} (varies by install).\\
\end{keytable}

\sect{Troubleshooting quick hits}\par
\begin{keytable}
\ks{F10 stolen} & Quit via menu (\ks{F9} $begin:math:text$\\rightarrow$end:math:text$ Quit) or remap the terminal key.\\
\ks{Alt combos} & Use \ks{Esc} then key (Meta emulation).\\
\ks{tmux/screen} & Try \ks{mc -g}; then re-run Learn keys.\\
\ks{Display odd} & Try \ks{mc -a} (ASCII) or adjust terminal settings.\\
\end{keytable}

\sect{References (URLs)}\par
\begin{keytable}
\ks{\manref{mc}{1}} & \tinyurl{https://source.midnight-commander.org/man/mc.html}\\
\ks{gist} & \tinyurl{https://gist.github.com/samiraguiar/9cd4264445545cfd459d}\\
\ks{site} & \tinyurl{https://midnight-commander.org/}\\
\end{keytable}

\end{multicols}

\end{document}

%%% Local Variables:
%%% mode: LaTeX
%%% TeX-master: t
%%% End:
